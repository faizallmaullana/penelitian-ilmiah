\documentclass[12pt]{article}
\usepackage[a4paper,top=4cm,bottom=3cm,left=4cm,right=3cm]{geometry}
\usepackage{setspace}
\usepackage{titlesec}
\usepackage{lipsum}
\usepackage{caption}
\usepackage{graphicx}
\usepackage{booktabs}
\usepackage{float}
\usepackage{tocloft}
\usepackage{indentfirst}

% Font Times New Roman
\usepackage{mathptmx}
% \usepackage{fontspec}
% \setmainfont{TeX Gyre Termes}

% Spasi teks utama
\onehalfspacing

% Judul Bab
\titleformat{\section}{\normalfont\bfseries\LARGE\MakeUppercase}{\thesection}{1em}{}
% Judul Sub-bab
\titleformat{\subsection}
  {\normalfont\bfseries\large}
  {\thesubsection}{1em}{}
% Indentasi paragraf pertama
\setlength{\parindent}{1.5em}
% Rata kiri-kanan
\renewcommand{\baselinestretch}{1.5}
\setlength{\parskip}{0pt plus 1pt}


\begin{document}

% =================== JUDUL ===================
\begin{center}
    \textbf{\LARGE Judul Penelitian}
\end{center}

% =================== ABSTRAK ===================
\begin{center}
    \textbf{\large ABSTRAK}
\end{center}
\singlespacing
\noindent
\fontsize{12pt}{14pt}\selectfont
\lipsum[1]

\vspace{1cm}
\onehalfspacing

% =================== DAFTAR ISI ===================
\renewcommand{\cftsecleader}{\cftdotfill{\cftdotsep}}
\singlespacing
\tableofcontents
\onehalfspacing
\newpage

% =================== BAB ===================
\section{Pendahuluan}
\lipsum[2-3]

\subsection{Latar Belakang}
\lipsum[4]

% =================== GAMBAR ===================
\begin{figure}[H]
    \centering
    \includegraphics[width=0.5\textwidth]{example-image}
    \caption{\textnormal{{Contoh Gambar: Proses Penelitian}}}
\end{figure}

% =================== TABEL ===================
\begin{table}[H]
    \centering
    \caption{\textnormal{{Contoh Tabel: Data Responden}}}
    \begin{tabular}{ll}
        \toprule
        Nama & Umur \\
        \midrule
        Ali & 23 \\
        Budi & 25 \\
        \bottomrule
    \end{tabular}
\end{table}

% =================== DAFTAR PUSTAKA ===================
\newpage
\singlespacing
\section*{Daftar Pustaka}
\begin{itemize}
    \item Doe, J. (2020). *Contoh Referensi*. Jakarta: Penerbit A.
    \item Smith, A. (2021). *Another Book*. Yogyakarta: Penerbit B.
\end{itemize}

% =================== DAFTAR LAMPIRAN ===================
\newpage
\section*{Lampiran}
\singlespacing
\noindent Lampiran 1: Kuesioner Penelitian

\end{document}
