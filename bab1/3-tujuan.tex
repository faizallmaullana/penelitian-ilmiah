\subsection{Tujuan}

Tujuan dari penelitian ini adalah untuk mengeksplorasi dan menganalisis penerapan Kubernetes, khususnya versi ringan K3s, dalam pengelolaan aplikasi berbasis \textit{microservices} yang berjalan pada lingkungan dengan sumber daya terbatas. Secara khusus, penelitian ini bertujuan untuk:

\begin{enumerate}
    \item Menilai efisiensi penggunaan K3s sebagai alternatif Kubernetes untuk aplikasi skala kecil dan lingkungan simulasi, dengan mempertimbangkan penggunaan sumber daya yang lebih rendah dibandingkan dengan Kubernetes full version.
    
    \item Menganalisis bagaimana Terraform dapat meningkatkan efisiensi dalam penyediaan dan pengelolaan infrastruktur Kubernetes atau K3s melalui prinsip \textit{Infrastructure as Code} (IaC), dengan fokus pada otomatisasi, konsistensi, dan pengelolaan yang lebih mudah.
    
    \item Mengidentifikasi keuntungan yang dapat diperoleh organisasi atau individu dengan mengadopsi K3s dan Terraform dalam konteks penyediaan infrastruktur yang lebih ringan, hemat sumber daya, dan mudah dikelola dalam lingkungan pengembangan atau startup.
    
    \item Menyusun perbandingan antara penggunaan Kubernetes full version dan K3s dari segi efisiensi sumber daya, performa, dan kemudahan dalam pengelolaan, untuk memberikan wawasan kepada organisasi yang ingin memilih solusi orkestrasi yang sesuai dengan kebutuhan mereka.
\end{enumerate}